% Search for all the places that say "PUT SOMETHING HERE".

\documentclass[11pt]{article}
\usepackage{amsmath,textcomp,amssymb,geometry,graphicx}

\def\Name{Alexander Tom (ID: 20964861)}  % Your name
\def\Login{ee122-ki} % Your login
\def\HomeworkNum{1}

\title{EE122--Fall 2013 --- Solutions to Homework \HomeworkNum}
\author{\Name, \texttt{\Login}}
\markboth{EE122--Fall 2013 Homework \HomeworkNum \Name}{EE122--Fall
    2013 Homework 3 \Name, \texttt{\Login}}
\pagestyle{myheadings}

\begin{document}
\maketitle
 
\section*{Q1.}
\begin{itemize}
    \item[(a.)]
        \begin{verbatim}
; <<>> DiG 9.8.1-P1 <<>> www.google.com
;; global options: +cmd
;; Got answer:
;; ->>HEADER<<- opcode: QUERY, status: NOERROR, id: 23171
;; flags: qr rd ra; QUERY: 1, ANSWER: 5, AUTHORITY: 4, ADDITIONAL: 4

;; QUESTION SECTION:
;www.google.com.			IN	A

;; ANSWER SECTION:
www.google.com.		268	IN	A	74.125.239.112
www.google.com.		268	IN	A	74.125.239.116
www.google.com.		268	IN	A	74.125.239.113
www.google.com.		268	IN	A	74.125.239.115
www.google.com.		268	IN	A	74.125.239.114

;; AUTHORITY SECTION:
google.com.		14484	IN	NS	ns1.google.com.
google.com.		14484	IN	NS	ns3.google.com.
google.com.		14484	IN	NS	ns2.google.com.
google.com.		14484	IN	NS	ns4.google.com.

;; ADDITIONAL SECTION:
ns2.google.com.		263358	IN	A	216.239.34.10
ns3.google.com.		264248	IN	A	216.239.36.10
ns1.google.com.		264248	IN	A	216.239.32.10
ns4.google.com.		263358	IN	A	216.239.38.10

;; Query time: 1 msec
;; SERVER: 128.32.112.21#53(128.32.112.21)
;; WHEN: Sun Dec  1 13:4
        \end{verbatim}
        The first field is the NAME which is the name of the domain in question
        being returned.  The second field is the TTL. The third field is the
        CLASS of the records being requested which in our case is IN for
        internet.  The forth field is the TYPE which determines the record type
        which in our case is A which stands for a mapping a domain name to an
        IPv4 address. The last field is the IP address.
    \item[(b.)]
        \begin{verbatim}

; <<>> DiG 9.8.3-P1 <<>> google.com +trace
;; global options: +cmd
.			413209	IN	NS	l.root-servers.net.
.			413209	IN	NS	j.root-servers.net.
.			413209	IN	NS	e.root-servers.net.
.			413209	IN	NS	g.root-servers.net.
.			413209	IN	NS	k.root-servers.net.
.			413209	IN	NS	d.root-servers.net.
.			413209	IN	NS	b.root-servers.net.
.			413209	IN	NS	i.root-servers.net.
.			413209	IN	NS	a.root-servers.net.
.			413209	IN	NS	h.root-servers.net.
.			413209	IN	NS	c.root-servers.net.
.			413209	IN	NS	f.root-servers.net.
.			413209	IN	NS	m.root-servers.net.
;; Received 344 bytes from 128.32.136.9#53(128.32.136.9) in 780 ms

com.			172800	IN	NS	k.gtld-servers.net.
com.			172800	IN	NS	f.gtld-servers.net.
com.			172800	IN	NS	g.gtld-servers.net.
com.			172800	IN	NS	h.gtld-servers.net.
com.			172800	IN	NS	l.gtld-servers.net.
com.			172800	IN	NS	a.gtld-servers.net.
com.			172800	IN	NS	d.gtld-servers.net.
com.			172800	IN	NS	c.gtld-servers.net.
com.			172800	IN	NS	m.gtld-servers.net.
com.			172800	IN	NS	i.gtld-servers.net.
com.			172800	IN	NS	j.gtld-servers.net.
com.			172800	IN	NS	b.gtld-servers.net.
com.			172800	IN	NS	e.gtld-servers.net.
;; Received 488 bytes from 192.112.36.4#53(192.112.36.4) in 488 ms

google.com.		172800	IN	NS	ns2.google.com.
google.com.		172800	IN	NS	ns1.google.com.
google.com.		172800	IN	NS	ns3.google.com.
google.com.		172800	IN	NS	ns4.google.com.
;; Received 164 bytes from 192.41.162.30#53(192.41.162.30) in 69 ms

google.com.		300	IN	A	74.125.239.100
google.com.		300	IN	A	74.125.239.104
google.com.		300	IN	A	74.125.239.99
google.com.		300	IN	A	74.125.239.110
google.com.		300	IN	A	74.125.239.105
google.com.		300	IN	A	74.125.239.102
google.com.		300	IN	A	74.125.239.103
google.com.		300	IN	A	74.125.239.96
google.com.		300	IN	A	74.125.239.97
google.com.		300	IN	A	74.125.239.101
google.com.		300	IN	A	74.125.239.98
;; Received 204 bytes from 216.239.34.10#53(216.239.34.10) in 64 ms
        \end{verbatim}
        Sequence of the name servers queried and domain each server was responsible for:
        \begin{verbatim}
        Name server, Domain Server resonsible for
        *.root-servers.net, .
        *.gtld-servers.net, com.
        ns*.google.com, google.com.
        74.125.239.*, google.com.
        \end{verbatim}

    \item[(c.)]
        DNS Query using ns1.ii.kgp.ac.in:

        \begin{verbatim}
; <<>> DiG 9.8.1-P1 <<>> www.google.com @ns1.iitkgp.ac.in
;; global options: +cmd
;; Got answer:
;; ->>HEADER<<- opcode: QUERY, status: NOERROR, id: 20675
;; flags: qr rd ra; QUERY: 1, ANSWER: 5, AUTHORITY: 4, ADDITIONAL: 1

;; QUESTION SECTION:
;www.google.com.			IN	A

;; ANSWER SECTION:
www.google.com.		100	IN	A	74.125.236.210
www.google.com.		100	IN	A	74.125.236.211
www.google.com.		100	IN	A	74.125.236.208
www.google.com.		100	IN	A	74.125.236.209
www.google.com.		100	IN	A	74.125.236.212

;; AUTHORITY SECTION:
google.com.		156744	IN	NS	ns4.google.com.
google.com.		156744	IN	NS	ns2.google.com.
oogle.com.		156744	IN	NS	ns1.google.com.
google.com.		156744	IN	NS	ns3.google.com.

;; ADDITIONAL SECTION:
ns1.google.com.		6138	IN	A	216.239.32.10

;; Query time: 292 msec
;; SERVER: 203.110.245.241#53(203.110.245.241)
;; WHEN: Sun Dec  1 14:28:20 2013
;; MSG SIZE  rcvd: 200
        \end{verbatim}

        DNS Query using nsl.fujitsu.fr:

        \begin{verbatim}
; <<>> DiG 9.8.1-P1 <<>> www.google.com @nsl.fujitsu.fr
;; global options: +cmd
;; Got answer:
;; ->>HEADER<<- opcode: QUERY, status: NOERROR, id: 59463
;; flags: qr rd ra; QUERY: 1, ANSWER: 5, AUTHORITY: 4, ADDITIONAL: 4

;; QUESTION SECTION:
;www.google.com.			IN	A

;; ANSWER SECTION:
www.google.com.		258	IN	A	173.194.40.180
www.google.com.		258	IN	A	173.194.40.176
www.google.com.		258	IN	A	173.194.40.177
www.google.com.		258	IN	A	173.194.40.178
www.google.com.		258	IN	A	173.194.40.179

;; AUTHORITY SECTION:
google.com.		308514	IN	NS	ns1.google.com.
google.com.		308514	IN	NS	ns2.google.com.
google.com.		308514	IN	NS	ns3.google.com.
google.com.		308514	IN	NS	ns4.google.com.

;; ADDITIONAL SECTION:
ns1.google.com.		62117	IN	A	216.239.32.10
ns2.google.com.		62117	IN	A	216.239.34.10
ns3.google.com.		62117	IN	A	216.239.36.10
ns4.google.com.		62117	IN	A	216.239.38.10

;; Query time: 165 msec
;; SERVER: 62.244.109.18#53(62.244.109.18)
;; WHEN: Sun Dec  1 14:29:27 2013
;; MSG SIZE  rcvd: 248
        \end{verbatim}
        We see a difference in latency because the latancy
            is directly propotional to the distance to the DNS server.
            nsl.fujitsu.fr is closer than ns1.iitkgp.ac.in. We can prove this
            by looking at the prefix of the ip address of the name servers.
            nsl.fujitsu.fr goes to a DNS server whose IP address is
            62.244.109.18 which is in France and ns1.iitkgp.ac.in maps to a DNS
            server whose IP address is 203.110.245.241 which is in India. When
            we dont specify a DNS server, the DNS server is 128.32.112.21 which
            is right in Berkeley.
    \item[(d.)]
        Garry's DNS server must return the following query to redirect traffic to his
        eavilsearch.com:

        \begin{verbatim}
For Type = A, the DNS server should return:
name = google.com
value = <IP address of evilsearch.com>
ttl = <ttl value less than the ttl to google.com>

For Type = NS, the DNS server should return:
name = google.com
value = <dns server of evilsearch.com>
ttl = <ttl value less than the ttl to google.com>
        \end{verbatim}
        To prevent this attack, the DNS server can look at
        the IP address of the DNS servers and verify they all match since
    google has multple dns servers. You can also ping multiple DNS servers and
verify that there is overlap between the results.
\end{itemize}
\newpage
 
 
\section*{Q2.}
\begin{itemize}
    \item[(a.)]
        \textbf{\underline{10R}}
    \item[(b.)]
        \textbf{\underline{6R}}
    \item[(c.)]
        \textbf{\underline{2R + 2 * 4/4R = 4R}}
    \item[(d.)]
        \textbf{\underline{2R + 2 * 4 * 0.5R = 6R}}
    \item[(e.)]
        \textbf{\underline{2R + 0.5R + 4 * 0.5R = 4.5R}}
    \item[(f.)]
        \textbf{\underline{2R + 2 * 4/4R * 0.5 = 3R}}
    \item[(g.)]
        \textbf{\underline{2R + 2 * 3/4R + 2 * 1/2R + 2 * 1/3R + 2 * 1/4R = 5.666R}}
    \item[(h.)]
        \textbf{\underline{2R + 2 * 3/4R + 2 * 1/2R + 2 * 1/3R + 2 * 1/4R = 5.666R}}
    \item[(i.)]
        \textbf{\underline{2R + max\{2 * 3/4R, 2 * 1/2R, 2 * 1/3R, 2 * 1/4R\} = 3.5R}}
\end{itemize}
\newpage
 
 
\section*{Q3.}
\begin{itemize}
    \item[(a.)]
        \begin{itemize}
            \item \textbf{Case 1:}
                \begin{itemize}
                    \item[(i.)]
                        Yes the transmission is successful and the original
                        transmission is NOT impacted. B can listen to more
                        than one node and B is not sending anything.
                    \item[(ii.)]
                        No, A cannot transmit to B because B is already
                        receiving from E. MACA will not send a CTS until
                        the transmission between E and B is done.
                \end{itemize}
            \item \textbf{Case 2:}
                \begin{itemize}
                    \item[(i.)]
                        Yes the transmission is successful and the original
                        transmission is NOT impacted.
                    \item[(ii.)]
                        Yes the transmission is successful and the original
                        transmission is NOT impacted.
                \end{itemize}
            \item \textbf{Case 3:}
                \begin{itemize}
                    \item[(i.)]
                        Yes the transmission is successful and the original
                        transmission is NOT impacted.
                    \item[(ii.)]
                        No, the transmission is not successful because A
                        overheard a CTS from B. 
                \end{itemize}
        \end{itemize}

    \item[(b.)]
        \begin{itemize}
            \item \textbf{Case 1:}
                \begin{itemize}
                    \item[(i.)]
                        No, A cannot transmit to B because B is already
                        transmitting to E.
                    \item[(ii.)]
                        No, A cannot transmit to B because B is already
                        sending to E. MACA will not send a CTS until
                        the transmission between E and B is done.
                \end{itemize}
            \item \textbf{Case 2:}
                \begin{itemize}
                    \item[(i.)]
                        Yes the transmission is successful and the original
                        transmission is NOT impacted.
                    \item[(ii.)]
                        No, F doesn't hear an CTS back from C because C
                        heard an CTS from B.
                \end{itemize}
            \item \textbf{Case 3:}
                \begin{itemize}
                    \item[(i.)]
                        No, C needs to wait for B to finish transmitting.
                    \item[(ii.)]
                        No, C and A don't hear a CTS. B is transmitting data and A won't hear an
                        RTS from C.
                \end{itemize}
            \item[(c.)]
                \begin{itemize}
                    \item[(i.)]
                        Nothing can successfully communicate. Every node is waiting for A to be done transmitting to B.
                    \item[(ii.)]
                        Nothing can successfully communicate. Every node is waiting for an RTS from A.
                \end{itemize}
            \item[(d.)]
                Using CS, yes, the nodes can successfully transmit because D, E, and F
                never hear someone else transmitting.  A, B, and C hear transmitting
                but that information doesn't propogate to D, E, F.
                \newline
                Using MACA, no, the nodes cannot transmit all at once because if any
                one of A, B, or C begins transmitting, the other two nodes overheard a
                CTS and have to wait for the transmission to end.
            \item[(e.)]
                Using CS, (D, A), (E, B), and (F, C) can simultaneously communicate.
                Each edge won't see each other transmitting, so both sides on the edge
                can begin transmitting simultaneously.
                \newline
                Using MACA, (D, A), (E, B), and (F, C) can simultaneously communicate.
                If both sides of the edge send a RTS at the same time, they will both
                send back a CTS and begin transmitting.
        \end{itemize}
    \end{itemize}
\newpage
 
 
\section*{Q4.}
\begin{itemize}
    \item[(a.)]
        \begin{itemize}
            \item[(i.)]
                \textbf{\underline{Root: 0; Edges: (4, 1), (1, 0), (0, 2), (2, 3), (3, 5)}}
            \item[(ii.)]
                \textbf{\underline{Root: 1; Edges: (1, 2), (2, 3), (1, 4), (4, 5)}}
        \end{itemize}
    \item[(b.)]
        \begin{itemize}
            \item[(1.)]
                \textbf{\underline{Switches: 0, 1, 2, 3, 4, 5; End-hosts: a, c, d, e, f, g}}
            \item[(2.)]
                \textbf{\underline{Switches: 2, 0, 1; End-hosts: b}}
            \item[(3.)]
                \textbf{\underline{Switchea: 2, 3, 5; End-hosts: c, f}}
            \item[(4.)]
                \textbf{\underline{Switches: 0, 1; End-hosts: b}}
            \item[(5.)]
                \textbf{\underline{Switches: 0, 1, 2, 3, 4, 5; End-hosts: b, c, d, e, f, g}}
        \end{itemize}
    \item[(c.)]
        \begin{itemize}
            \item[(1.)]
                \textbf{\underline{Flood}}
            \item[(2.)]
                \textbf{\underline{Unicast}}
            \item[(3.)]
                \textbf{\underline{Flood}}
            \item[(4.)]
                \textbf{\underline{Unicast}}
            \item[(5.)]
                \textbf{\underline{Unicast}}
            \item[(6.)]
                \textbf{\underline{Flood}}
            \item[(7.)]
                \textbf{\underline{Unicast}}
            \item[(8.)]
                \textbf{\underline{Unicast}}
            \item[(9.)]
                \textbf{\underline{Flood}}
            \item[(10.)]
                \textbf{\underline{Unicast}}
            \item[(11.)]
                \textbf{\underline{Unicast}}
            \item[(12.)]
                \textbf{\underline{Flood}}
        \end{itemize}
        \begin{itemize}
            \item[(i.)]
                (a to b): 0/4 flooded, 4/4 Unicasted; (b to c): 1/4 flooded, 3/4 Unicasted; (c to b): 4/4 Flooded, 0/4 Unicasted
            \item[(ii.)]
                Swap 8 and 9 and swap 11 and 12
        \end{itemize}
\end{itemize}
\end{document}
